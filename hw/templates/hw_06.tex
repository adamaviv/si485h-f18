\documentclass{article}[9pt]

\usepackage{listings}
\usepackage{fullpage}
\usepackage{textcomp}

\lstset{ %
  basicstyle=\ttfamily\small,
  commentstyle=\ttfamily\small\emph,
  upquote=true,
  framerule=1.25pt,
  breaklines=true,
  showstringspaces=false,
  escapeinside={(*@}{@*)},
  belowskip=2em,
  aboveskip=1em,
}


\newenvironment{answerfont}{\fontfamily{qhv}\selectfont}{\par}
\newenvironment{myanswer}{\begin{mdframed}\begin{answerfont}}{\end{answerfont}\end{mdframed}}


%%%%%%%%%%%%%%%%%%%%%%%%%%
% To add an answer, use the myanswer environment command under \item heading,like
%
% \item What is the meaning of life, universe, everything?
% \begin{mysanswer}
%   The answer is 42!
% \end{myanswer}
%
% You can also use macro's in your \myanswer if you want to control the
% formatting. Such as verbatim enviroment or lstlisting environment.
%
% BUT(!) I have to be able to read/grade your resulting pdf to give you credit,
% so make sure your home work is well formatted.
%%%%%%%%%%%%%%%%%%%%%%%%%%%

\title{SI485h: HW 06}
\date{FILL IN THE DUE DATE}
\author{FILL IN YOUR NAME}

\begin{document}

\maketitle
\section*{Instructions}

\begin{itemize}
\item You must submit your homework using this Latex template.
  
\item This homework is graded out of 70 points. Point values are associated to each question.
\end{itemize}

\section*{Questions}
\label{sec:org24d21ef}

\begin{enumerate}
\item (4 points) How many bits in a 32-bit memory address are typically randomized
in ASLR?

\item For the following addresses, what is the non random base address?
For example, for the stack, the base portion is 0xbf8000.

\begin{enumerate}
\item (2 points) \texttt{0x09bed40}

\item (2 points) \texttt{0xbd90023}

\item (2 points) \texttt{0xe11eefa}
\end{enumerate}

\item (10 point) Consider a guessing game where you are trying to guess the number
I'm thinking of right now. It's a number between 1 and m. Every
time you guess, I change my number! How many guesses, in terms of
n, would it take until you had a 50\% chance of getting it right
within that number of guesses? 

Write your answer in reduced form with respect to m and n. 

(\emph{hint}: Use base 2 log is key!)

\item (5 points) Explain how the above problem related to breaking ASLR.

\item (5 points) Explain how a NOP sled improves your ability to brute force an
ASLR program? Consider the situation of 32-bit address space.

\item Consider a \textbf{48 bit} memory address space (6 bytes) that employs
22 bits of randomness when basing the stack address. The first 20
bits are fixed for the stack, e.g., 0xbff, the next 22 bits are
random, and the last 6 bits are fixed. (\emph{hint: you may want to
write a program to solve the questions below.})

\begin{enumerate}
\item (5 points) If you had no NOP sled, how many brute force attempts would it
take to exploit this program with 50\% likelihood.

\item (5 points) If you had a NOP sled of 255 bytes long, how many brute force
attempts would it take to exploit this program with 50\%
likelihood?

\item (5 points) If you had a NOP sled of 8192 bytes long, how many brute force
attempts would it take to exploit this program with 50\%
likelihood?

\item (5 points) Suppose you wanted to ensure you exploited this program with
50\% likelihood with 25 guesses, how long must your NOP sled
be?
\end{enumerate}

\item Complete the stack diagram for leave and ret for the following
function frame:

\begin{verbatim}
      .----------------.
      | shell code     |
      |----------------|
      | return address |
      |----------------|
ebp-> |       SBP      |
      |----------------|
      |                |
      :                :
      :                :
esp-> |                |
      '----------------'
\end{verbatim}

\begin{enumerate}
\item (5 points) After the leave

\item (5 points) After the ret
\end{enumerate}

\item (5 points) Using the diagrams from the previous question in your answer,
explain why \texttt{jmp esp} or \texttt{call esp} instructions will execute
your shell code? How can this technique be used to subvert ASLR?

\item (5 points) The following is a hexdump of the text segment of a
program. What addresses (plural!) could be used as bounce
points?

\begin{verbatim}
0x08048470 b830a004 082d30a0 0408c1f8 0289c2c1 .0...-0.........
0x08048480 ea1f01d0 d1f87501 c3ba0000 000085d2 ......u.........
0x08048490 74f65589 e583ec18 89442404 c7042430 t.U......D$...$0
0x080484a0 a00408ff d2c9c389 f68dbc27 00000000 ...........'....
0x080484b0 803d30a0 04080075 135589e5 83ec08e8 .=0....u.U......
0x080484c0 7cffffff c60530a0 040801c9 f3c36690 |.....0.......f.
0x080484d0 a1109f04 0885c074 1fb80000 000085c0 .......t........
0x080484e0 74165589 e583ec18 c7042410 9f0408ff t.U.......$.....
0x080484f0 d0c9e979 ffffff90 e973ffff ff5589e5 ...y.....s...U..
0x08048500 5dc35589 e583ec18 c7042470 860408e8 ].U.......$p....
0x08048510 acfeffff ffe4c9c3 5589e583 ec18c704 ........U.......
0x08048520 24858604 08e896fe ffffffd4 c9c35589 $.............U.
0x08048530 e583ec48 8b450c89 45c465a1 14000000 ...H.E..E.e.....
0x08048540 8945f431 c0c745d0 00000000 8b45c489 .E.1..E......E..
0x08048550 4424048d 45d48904 24e852fe ffffeb20 D$..E...$.R....      
\end{verbatim}
\end{enumerate}
\end{document}