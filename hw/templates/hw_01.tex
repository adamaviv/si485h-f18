\documentclass{article}[9pt]

\usepackage{listings}
\usepackage{fullpage}
\usepackage{textcomp}
\usepackage{mdframed}

\lstset{ %
  basicstyle=\ttfamily\small,
  commentstyle=\ttfamily\small\emph,
  upquote=true,
  framerule=1.25pt,
  breaklines=true,
  showstringspaces=false,
  escapeinside={(*@}{@*)},
  belowskip=2em,
  aboveskip=1em,
}


\newenvironment{answerfont}{\fontfamily{qhv}\selectfont}{\par}
\newenvironment{myanswer}{\begin{mdframed}\begin{answerfont}}{\end{answerfont}\end{mdframed}}


%%%%%%%%%%%%%%%%%%%%%%%%%%
% To add an answer, use the myanswer environment command under \item heading,like
%
% \item What is the meaning of life, universe, everything?
% \begin{myanswer}
%   The answer is 42!
% \end{myanswer}
%
% You can also use macro's in your \myanswer if you want to control the
% formatting. Such as verbatim environment or lstlisting environment.
%
% BUT(!) I have to be able to read/grade your resulting pdf to give you credit,
% so make sure your home work is well formatted.
%%%%%%%%%%%%%%%%%%%%%%%%%%%

\title{SI485h: HW 01}
\date{FILL IN THE DUE DATE}
\author{FILL IN YOUR NAME}

\begin{document}

\maketitle
\section*{Instructions}

\begin{itemize}
\item You must submit your homework using this Latex template.
  
\item This homework is graded out of 100 points. Point values are associated to each question
\end{itemize}

\section*{Questions}

\begin{enumerate}
\item (10-points) For the following C program below, label the different parts of it:
\begin{verbatim}
#include <stdio.h>
#include <stdlib.h>
int main(int argc, char * argv[]){
  printf("Hello World!\n");

  return 0;
}
\end{verbatim}
\begin{enumerate}
\item number of command line arguments

\item library functions

\item return value

\item header files
\end{enumerate}

\item (5 points) Assume you got the following compiler error:
\begin{verbatim}
user@si485H-base:demo$ gcc -o hello hello.c /tmp/ccC4VYbK.o:
In function `main': hello.c:(.text+0x20): undefined reference to `world' collect2: error: ld returned 1 exit status
\end{verbatim}

\begin{enumerate}
\item Explain this compiler error

\item what would the programmer need to provide in order to complete the compilation?
\end{enumerate}

\item (5 points) What is the difference between a system call and a library call?
Provide a concrete example of each (one library call and one
system call) in your answer.

\item (15 points) Clone the following git repo:
\begin{verbatim}
git clone git@si485h-git.academy.usna.edu:aviv/HW-1.git
\end{verbatim}
Where you will find \texttt{trace-me-1}:

\begin{enumerate}
\item In \texttt{trace-me-1}: What are the library functions used in this program as reported by \texttt{ltrace}

\item In \texttt{trace-me-1}: Match the system calls, via \texttt{strace} (or \texttt{ltrace}!), to the library calls.
\end{enumerate}

\item (20 points) In the same repo, you'll find another program,
\texttt{trace-me-2} that has a secret message embedded within it. Using
\texttt{ltrace} and \texttt{strace}, extract the secret message, and explain
how you did so. If you were unable to find the secret message,
describe your process for partial credit.

\item (10 points) Convert the four-byte hex values into their signed and unsigned
values (Yes! You could use a computer for this!).

\begin{enumerate}
\item \texttt{0xffffffff}

\item \texttt{0x0000000b}

\item \texttt{0x8000000b}

\item \texttt{0xdeadbeef}
\end{enumerate}

\item (10 points) Complete the memory model for the following program at each of the labeled marks.
\begin{verbatim}
#include <stdio.h>
#include <stdlib.h>

int main(int argc, char * argv[]){
  int a[] = {10,11,12,13};
  int * p = a; //MARK 0

  p++; //MARK 1

  p[1] = 50;
  p--; //MARK 2

  *p = 12; //MARK 3
}
\end{verbatim}

Diagram at MARK 0
\begin{verbatim}
     .------.
a--> |  10  | <--.
     |------|    |
     |  11  |    |
     |------|    |
     |  12  |    |
     |------|    |
     |  13  |    | 
     |------|    | 
  p  |   .--+----'         
     '------'
\end{verbatim}

\item (10 points) Fill in the memory diagram for array values byte-by-byte. Pay
careful attention to the memory addresses in the diagram.

\begin{verbatim}
#include <stdio.h>
#include <stdlib.h>

int main(int argc, char * argv[]){
  int a[] = {0xcafebabe,0xdeadbeef};
  char * p = (char *) a;

  p++;
  *p = 0x00; //MARK
}

\end{verbatim}

Diagram to be completed
\begin{verbatim}

   a == 0xbffff6b4

   addr       value
.------------------.
| 0xbffff6b4 |     |
| 0xbffff6b5 |     |
| 0xbffff6b6 |     |
| 0xbffff6b7 |     |
| 0xbffff6b8 |     |
| 0xbffff6b9 |     |
| 0xbffff6ba |     |
| 0xbffff6bb |     |
'------------'-----. 
\end{verbatim}

\item (15 points) Consider the following program, label the place in memory where
each of the variable's values are stored. The answers can be
\texttt{reserved}, \texttt{stack}, \texttt{heap}, \texttt{text}, or \texttt{bss}.

\begin{verbatim}
#include <stdio.h>
#include <stdlib.h>

int mystrlen(int *s){
  int i;
  for(i=0;*s++;i++);
  return i;
}

int main(int argc, char * argv[]){
  char a[] = "hello";
  char * b = getenv("PATH");
  char * c = "world";
}

\end{verbatim}

\begin{enumerate}
\item \texttt{\&a}

\item \texttt{b}

\item \texttt{c}

\item \texttt{mystrlen}

\item \texttt{s}
\end{enumerate}
\end{enumerate}
\end{document}
%%% Local Variables:
%%% mode: latex
%%% TeX-master: t
%%% End:
