\documentclass{article}[9pt]

\usepackage{listings}
\usepackage{fullpage}
\usepackage{textcomp}
\usepackage{mdframed}

\lstset{ %
  basicstyle=\ttfamily\small,
  commentstyle=\ttfamily\small\emph,
  upquote=true,
  framerule=1.25pt,
  breaklines=true,
  showstringspaces=false,
  escapeinside={(*@}{@*)},
  belowskip=2em,
  aboveskip=1em,
}


\newenvironment{answerfont}{\fontfamily{qhv}\selectfont}{\par}
\newenvironment{myanswer}{\begin{mdframed}\begin{answerfont}}{\end{answerfont}\end{mdframed}}


%%%%%%%%%%%%%%%%%%%%%%%%%%
% To add an answer, use the myanswer environment command under \item heading,like
%
% \item What is the meaning of life, universe, everything?
% \begin{mysanswer}
%   The answer is 42!
% \end{myanswer}
%
% You can also use macro's in your \myanswer if you want to control the
% formatting. Such as verbatim enviroment or lstlisting environment.
%
% BUT(!) I have to be able to read/grade your resulting pdf to give you credit,
% so make sure your home work is well formatted.
%%%%%%%%%%%%%%%%%%%%%%%%%%%

\title{SI485h: HW 02}
\date{FILL IN THE DUE DATE}
\author{FILL IN YOUR NAME}

\begin{document}

\maketitle

\section*{Instructions}

\begin{itemize}
\item You must submit your homework using this Latex template.
  
\item This homework is graded out of 125 points. Point values are associated to each question
\end{itemize}

\section*{Questions}

\begin{enumerate}
\item (15 Points) Consider the following parts of the disassembled code below. For
each labeled part, provide a short, plain-English description.

\begin{verbatim}
.--------------------------------------.
| 0x0804841d <+0>: push ebp            |
| 0x0804841e <+1>: mov ebp,esp         | (a)
| 0x08048420 <+3>: and esp,0xfffffff0  |
'--------------------------------------'
.---------------------------------.
| 0x08048423 <+6>: sub esp,0x30   | (b)
'---------------------------------'

.--------------------------------------------------------.
| 0x08048426 <+9>: mov DWORD PTR [esp+0x1d],0x6c6c6548   |
| 0x0804842e <+17>: mov DWORD PTR [esp+0x21],0x57202c6f  |
| 0x08048436 <+25>: mov DWORD PTR [esp+0x25],0x646c726f  | (c)
| 0x0804843e <+33>: mov WORD PTR [esp+0x29],0xa21        |
| 0x08048445 <+40>: mov BYTE PTR [esp+0x2b],0x0          |
'--------------------------------------------------------'
.------------------------------------------------.
| 0x0804844a <+45>: lea eax,[esp+0x1d]           |
| 0x0804844e <+49>: mov DWORD PTR [esp+0x2c],eax | (d)
'------------------------------------------------'
.---------------------------------------------.
| 0x08048452 <+53>: jmp 0x804846b <main+78>   | (e)
'---------------------------------------------'

.--------------------------------------------------.
| 0x08048454 <+55>: mov eax,DWORD PTR [esp+0x2c]   |
| 0x08048458 <+59>: movzx eax,BYTE PTR [eax]       |
| 0x0804845b <+62>: movsx eax,al                   |
| 0x0804845e <+65>: mov DWORD PTR [esp],eax        | (f)
| 0x08048461 <+68>: call 0x8048310 <putchar@plt>   |
| 0x08048466 <+73>: add DWORD PTR [esp+0x2c],0x1   |
'--------------------------------------------------'
.--------------------------------------------------.
| 0x0804846b <+78>: mov eax,DWORD PTR [esp+0x2c]   |
| 0x0804846f <+82>: movzx eax,BYTE PTR [eax]       |
| 0x08048472 <+85>: test al,al                     | (g)
| 0x08048474 <+87>: jne 0x8048454 <main+55>        |
| 0x08048476 <+89>: mov eax,0x0                    |
'--------------------------------------------------'
.---------------------------.
| 0x0804847b <+94>: leave   |
|0x0804847c <+95>: ret      | (h)
'---------------------------'
\end{verbatim}

\item (5 points) What size data element does each of the following \texttt{PTR} reference?

\begin{enumerate}
\item \texttt{BYTE PTR}

\item \texttt{DWORD PTR}

\item \texttt{WORD PTR}
\end{enumerate}

\item (15 points) Consider the disassembled program below:

\begin{verbatim}
0x0804841d <+0>: push ebp
0x0804841e <+1>: mov ebp,esp
0x08048420 <+3>: and esp,0xfffffff0
0x08048423 <+6>: sub esp,0x30
0x08048426 <+9>: mov DWORD PTR [esp+0x1f],0x74616542
0x0804842e <+17>: mov DWORD PTR [esp+0x23],0x796d7241
0x08048436 <+25>: mov BYTE PTR [esp+0x27],0x0
0x0804843b <+30>: mov DWORD PTR [esp+0x2c],0x0
0x08048443 <+38>: jmp 0x804846c <main+79>
0x08048445 <+40>: mov DWORD PTR [esp+0x28],0x0
0x0804844d <+48>: jmp 0x8048460 <main+67>
0x0804844f <+50>: lea eax,[esp+0x1f]
0x08048453 <+54>: mov DWORD PTR [esp],eax
0x08048456 <+57>: call 0x80482f0 <puts@plt>
0x0804845b <+62>: add DWORD PTR [esp+0x28],0x1
0x08048460 <+67>: cmp DWORD PTR [esp+0x28],0x3
0x08048465 <+72>: jle 0x804844f <main+50>
0x08048467 <+74>: add DWORD PTR [esp+0x2c],0x1
0x0804846c <+79>: cmp DWORD PTR [esp+0x2c],0x4
0x08048471 <+84>: jle 0x8048445 <main+40>
0x08048473 <+86>: leave
0x08048474 <+87>: ret 
\end{verbatim}

\begin{itemize}
\item How many times is \texttt{puts()} called? Explain.

\item What is being outputted for each \texttt{puts()} call? Explain.

\item Write the C equilvanent of this code.
\end{itemize}

\item (15 point) Consider the following scenario for function a function \texttt{bar()}
about call a function \texttt{foo()}: the calling function \texttt{bar()}'s
instruction pointer will be at address \texttt{0x8045520} after \texttt{foo()}
returns; the register \texttt{ebp} has value \texttt{0xbfff540}; The register
\texttt{esp} has value \texttt{xbffff504}; \texttt{foo()} takes one argument, an
integer with value \texttt{0xdeadbeef}.

We can then begin to trace the stack frame like so:
\begin{verbatim}
                                   call <foo>
       .------------.           .------------.           
 ebp-> |            |      ebp->|            |
       |            |           |            |
       |            |           |            |
esp -> | 0xdeadbeef |           | 0xdeadbeef |
       '------------'      esp->| 0x08045520 |
                                '------------'
\end{verbatim}
 Complete the remaining stack frames after each of the following
instructions complete.

\begin{itemize}
\item \texttt{push ebp}
\item \texttt{mov ebp,esp}
\item \texttt{sub esp, 0x30}
\end{itemize}

\item (5 points) Using the stack from the previous question, consider what
happens when \texttt{foo()} returns. In x86, write the sequence of
instructions needed to reset the stack frame to \texttt{bar()}'s
execution. That is, you should have some set of \texttt{mov}'s \texttt{sub}'s
or \texttt{pop}'s or whatever other instruction you might need.

\item (5 points) If the register \texttt{eax} stores \texttt{0xdeadbeef}, what does the
following registers store?

\begin{itemize}
\item \texttt{ax}
\item \texttt{al}
\item \texttt{ah}
\end{itemize}

\item (10 points) Consider the following source code below and disassembly, match
the memory address to the variable name.

\begin{verbatim}
int foo(int a){
  int i,r;
  r=0;
  for(i=1;i<=a;i++){
    r+=i;
  }
  return r;
}
\end{verbatim}
\begin{verbatim}
0x0804846d <+0>: push ebp
0x0804846e <+1>: mov ebp,esp
0x08048470 <+3>: sub esp,0x10
0x08048473 <+6>: mov DWORD PTR [ebp-0x4],0x0
0x0804847a <+13>: mov DWORD PTR [ebp-0x8],0x1
0x08048481 <+20>: jmp 0x804848d <foo+32>
0x08048483 <+22>: mov eax,DWORD PTR [ebp-0x8]
0x08048486 <+25>: add DWORD PTR [ebp-0x4],eax
0x08048489 <+28>: add DWORD PTR [ebp-0x8],0x1
0x0804848d <+32>: mov eax,DWORD PTR [ebp-0x8]
0x08048490 <+35>: cmp eax,DWORD PTR [ebp+0x8]
0x08048493 <+38>: jle 0x8048483 <foo+22>
0x08048495 <+40>: mov eax,DWORD PTR [ebp-0x4]
0x08048498 <+43>: leave
0x08048499 <+44>: ret 
\end{verbatim}

\begin{itemize}
\item \texttt{ebp-0x4}
\item \texttt{ebp-0x8}
\item \texttt{ebp+0x8}
\end{itemize}

\item (10 points) Clone the repository below here you will find a program called \texttt{main}.
\begin{verbatim}
git clone git@saddleback.academy.usna.edu:aviv/HW-2.git
\end{verbatim}
Execute the program under \texttt{gdb} and place a breakpoint at \texttt{foo},
and then run the program with the following arguments:
\begin{verbatim}
(gdb) r 5 1
\end{verbatim}

\begin{itemize}
\item Fill in the missing pieces of the memory diagram \emph{after} the
instruction at \texttt{0x08048448} executes.
\begin{verbatim}
     .------------------.
     |                  |
     |------------------|
     |                  |
     |------------------|
     |                  |
     |------------------|
ebp->|    0xbffff658    |
     |------------------|
     |                  | 
     |------------------|
esp->|                  |
     '------------------'
\end{verbatim}

\item Fill in the missing pieces of the memory diagram \emph{after} the
instruction at \texttt{0x08048460} executes for \emph{second} time.

\begin{verbatim}
     .------------------.
     |                  |
     |------------------|
     |                  |
     |------------------|
     |                  |
     |------------------|
ebp->|    0xbffff658    |
     |------------------|
     |                  | 
     |------------------|
esp->|                  |
     '------------------'
\end{verbatim}
\end{itemize}

\item (5 points) In the HW2 git repo you cloned before, you will find a program
called \texttt{stacked}. Run this program under \texttt{gdb} and place a break
point at the function \texttt{baz()}.

\begin{itemize}
\item How many functions are called before \texttt{baz()}?
\item How did you determine that?
\end{itemize}

\item (20 points) In the HW2 git repo you cloned before, you will find a program
called \texttt{trace-me}. Using \texttt{gdb} determine the right input for
the program such that it prints the secret message.

\begin{itemize}
\item What is the secret message?
\item How did you determine the right input?
\end{itemize}

(Hint 1: Does it take arguemnts?)

(Hint 2: Trace main using ni [next instruction])

(Hint 3: What is being compared?)

\item (20 points) In the HW3 git repo you cloned before, you will find a program
called \texttt{crack-me}. Use \texttt{gdb} to inspect the various elements of
the \texttt{main()} function to determine the secret message.

\begin{itemize}
\item What is the secret message?
\item How did you determine the right input?
\end{itemize}

(Hint 1: use examine [i.e., x] check things out)

(Hint 2: use print to perform operations and see results)

(Hint 3: use print again to reinterpret those results)
\end{enumerate}
\end{document}